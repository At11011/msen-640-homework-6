\input{./src/main.sty}

\begin{document}

\input{./src/titlepage.tex}

\pagebreak

\begin{enumerate}
  \item After careful synthesis, you finally manage to create a new
    substance (call it ‘A’) that maintains a
    liquid phase over a wide range of temperatures and will mix with
    water on warm days but is immiscible
    with water on colder days. The mixture of A and water seems to
    obey the regular solution model to
    an absurd degree. Before publishing your results, you hope to
    characterize binary mixtures of A and
    water in terms of the regular solution model through a series of
    experiments.

    \begin{enumerate}[(a.)]
      \item In preparation, you wish to recall the relationship
        between equilibrium volume fraction and
        the interaction parameter $\chi$. Starting from the mixing free
        energy, derive this relationship.

        \boxedanswer{
          Recall the Gibbs free energy of mixing:

          \begin{equation}
            \Delta G_{\text{mix}} = a_0X_1X_2 + RT(X_1\ln X_1 + X_2
            \ln X_2) \tag{8.110}
          \end{equation}

          Set $X_1 = \phi_A$ and $a_0 = \chi RT$. Note that $X_2 = 1
          - X_1$, and that $\frac{d\Delta G_{\text{mix}}}{dX_1}$

          \begin{align*}
            \frac{d(\Delta G_{\text{mix}}/(RT))}{d\phi_A} &= \chi(1 -
            2\phi_A) + (\ln \phi_A + 1 - \ln (1 - \phi_A) + 1) = 0  \\
            \chi(2\phi_A - 1) &= RT\left(\ln \left[\frac{\phi_A}{1 -
            \phi_A}\right] + 2\right)  \\
            \Aboxed{\chi &= \frac{1}{2\phi_A - 1}\left(\ln
            \left[\frac{\phi_A}{1 - \phi_A}\right]\right)}
          \end{align*}
        }

      \item  Without bothering to measure the initial volume fraction
        of A, you start by cooling the mixture to $T =
        \SI{50}{\degree F}$, observing phase
        separation. Carefully
        drawing a sample from the A-rich phase, you
        then determine that it has a volume fraction of $\phi_A = 0.8$.
        What is the value of $\chi$ that this corresponds
        to?

        \boxedanswer{

          Plug into the above equation:

          \begin{align*}
            \chi &= \frac{1}{2\phi_A - 1}\left(\ln
            \left[\frac{\phi_A}{1 - \phi_A}\right] + 2\right) \\
            \chi &= \frac{1}{2(0.8) - 1}\left(\ln \left[\frac{0.8}{1
            - 0.8}\right]\right) \\
            \Aboxed{\chi &\approx 2.310}
          \end{align*}
        }

      \item In a separate experiment, you prepare a mixture, this
        time taking care to record a volume
        fraction $\phi_A^{(0)} = 0.6$, and place it in your lab
        refrigerator, cooling it to $T = \SI{35}{\degree F}$ . You
        notice an \underline{almost complete separation} of A from water.
        Thinking that you
        might have difficulty measuring the
        possibly trace amounts of water in the A-rich phase, you
        instead note that the A-rich phase occupies
        about \SI{62}{\percent} of the container. Use this
        information to determine
        \si{\percent} the volume fraction of the A-rich phase.

        \boxedanswer{

          We can assume that volume is conserved, and calculate the
          volume fraction as follows:

          \begin{align*}
            \phi_A^{(0)} &= a\phi_A^{(1)} + (1 - a)\phi_A^{(2)} \\
            &
            \begin{aligned}
              \phi_A^{(2)} \approx 0 \\
              a &= 0.62 \\
            \end{aligned} \\
            \phi_A^{(1)} &= \frac{0.6}{0.62} \\
            \Aboxed{\phi_A^{(1)} &= 0.968}
          \end{align*}
        }

      \item The interaction parameter can be expressed as a function
        of temperature as $\chi = \frac{A}{T} + B$,
        where $T$ is measured in Kelvin and $A$ and $B$ are material
        constants. Using the data obtained from your two
        experiments, determine $A$ and $B$.

        \boxedanswer{

          We have two data points, at the first data point, the interaction
          parameter is 5.644 at \SI{50}{\degree F} or \SI{283.15}{\kelvin},
          and the volume fraction is 0.968 at \SI{35}{\degree F} or
          \SI{274.816}{\kelvin}.
          This can be used to find the interaction parameter at this
          temperature,
          and then the linear equation can be fitted.

          \begin{align*}
            \chi_1 &= \frac{A}{T_1} + B \\
            \chi_2 &= \frac{A}{T_2} + B \\
            B &= \chi_1 - \frac{A}{T_1} \\
            A &= \frac{\chi_2 - \chi_1}{\left(\frac{1}{T_2}  -
            \frac{1}{T_1}\right)} \\
            % B &= \chi_1 - \frac{\chi_2 -
            % \chi_1}{\left(\frac{T_1}{T_2}  - 1\right)} \\
            A &= \frac{\chi_2 - \chi_1}{\left(\frac{1}{T_2}  -
            \frac{1}{T_1}\right)} \\
            &
            \begin{aligned}
              \chi_1 &= 2.31 \\
              \chi_2 &= \frac{1}{2\phi_A - 1}\left(\ln
              \left[\frac{\phi_A}{1 - \phi_A}\right]\right) \\
              \chi &= \frac{1}{2(0.968) - 1}\left(\ln
              \left[\frac{(0.968)}{1 - 0.968}\right]\right) \\
              \chi_2 &= 3.66 \\
              T_1 &= \SI{283.15}{\kelvin} \\
              T_2 &= \SI{274.816}{\kelvin} \\
            \end{aligned} \\
            A &= \frac{(3.66) -
            (2.31)}{\left(\frac{1}{(\SI{283.15}{\kelvin})}  -
            \frac{1}{\SI{274.816}{\kelvin}}\right)} \\
            A &= \SI{12375.02}{\kelvin} \\
            B &= (2.31) -
            \frac{(\SI{12375.02}{\kelvin})}{(\SI{283.15}{\kelvin})} \\
            B &= -41.39 \\
            \Aboxed{\chi(T) &= \frac{\SI{12375.02}{\kelvin}}{T} -41.39}
          \end{align*}
        }

      \item Using what you have learned, determine the
        \textit{upper critical
        solution temperature (UCST)} – the
        temperature above which A and water are miscible for all
        compositions.

        \boxedanswer{

          The critical value for $\chi$ is 2. So:

          \begin{align*}
            \chi_c &= \frac{A}{T_c} + B \\
            T_c &= \left[\frac{\chi_c - B}{A}\right]^{-1} \\
            T_c &= \left[\frac{2 +
            41.39}{\SI{12375.02}{\kelvin}}\right]^{-1} \\
            \Aboxed{T_c &= \SI{285.18}{\kelvin}}
          \end{align*}
        }

    \end{enumerate}

    \pagebreak

  \item You have found that the partial molar volume of component 1
    of a two-component system is given by

    \begin{equation}
      \Delta \overline V_1 = 2aX_1X_2^2
    \end{equation}

    where $a = \SI{2.7166}{\centi\meter\cubed\per\mole}$

    \begin{enumerate}[(a.)]
      \item Determine an equation for the change in partial molar
        volume of component 2 during mixing
        as a function of composition.

        \boxedanswer{
          Use the Gibbs-Duhem relation:

          \begin{align*}
            X_1 d\Delta \overline{V}_1 + X_2 d\Delta
            \overline V_2 &= 0 \\
            d\Delta \overline V_2 &= -\frac{X_1}{X_2}d\Delta
            \overline{V}_1 \\
            &
            \begin{aligned}
              X_2 &= 1 - X_1 \\
              \Delta \overline V_1 &= 2 a X_1 (1 - X_1)^2 \\
              \Delta \overline V_1 &= 2 a (X_1  - 2X_1^2 + X_1^3) \\
              d\Delta \overline V_1 &= 2 a (3X_1^2 - 4X_1 + 1)dX_1 \\
              d\Delta \overline V_1 &= 2 a (3X_1 - 1)(X_1 - 1)dX_1 \\
            \end{aligned} \\
            d\Delta \overline V_2 &= -\frac{X_1}{\cancel{1-  X_1}}2 a
            (3X_1 - 1)(\cancel{X_1 - 1})dX_1 \\
            d\Delta \overline V_2 &= 2aX_1(3X_1 - 1)dX_1 \\
            \int d\Delta \overline V_2 &= \int 2aX_1(3X_1 - 1)dX_1 \\
            \Delta \overline V_2(X_1) &= 2a\left[X_1^3 -
            \frac{X_1^2}{2}\right] + C \\
            &
            \begin{aligned}
              \Delta \overline V_2(0) &= 0 \\
              C &= 0 \\
            \end{aligned} \\
            \Aboxed{\Delta \overline V_2(X_1) &= 2aX_1^3 - aX_1^2}
          \end{align*}
        }

      \item Determine an equation for the total volume change during
        mixing as a function of composition.

        \boxedanswer{
          The total volume change can be written:

          \begin{align*}
            \Delta \overline V_{\text{mix}} &= X_1 \Delta \overline
            V_1 + X_2 \Delta \overline V_2 \\
            &
            \begin{aligned}
              \Delta \overline V_1 &= 2aX_1 (1 - X_1)^2 \\
              \Delta \overline V_2(X_1) &= 2aX_1^3 - aX_1^2 \\
              X_2 &= 1 - X_1 \\
            \end{aligned} \\
            \Delta \overline V_{\text{mix}} &= X_12aX_1 (1 - X_1)^2 +
            (1 - X_1)(2aX_1^3 - aX_1^2) \\
            \Delta \overline V_{\text{mix}} &= 2aX_1^2 - 4aX_1^3 +
            \cancel{2aX_1^4} + 2aX_1^3 - aX_1^2 -
            \cancel{2aX_1^4} + aX_1^3 \\
            \Aboxed{\Delta \overline V_{\text{mix}} &= aX_1^2 - aX_1^3}
          \end{align*}
        }

      \item Calculate the total volume change of the mixture, the
        partial molar volume change of component 1, and the partial
        molar volume change of component 2
        (three separate numerical answers) when
        \SI{75}{\mole} of component 1 are mixed with \SI{25}{\mole}
        of component 2.

        \boxedanswer{
          \begin{align*}
            \Delta \overline V_{\text{mix}} &= aX_1^2 - aX_1^3 \\
            &
            \begin{aligned}
              X_1 &= \frac{75}{75 + 25} = 0.75 \\
              a &= \SI{2.7166}{\centi\meter\cubed\per\mole}
            \end{aligned} \\
            \Delta \overline V_{\text{mix}} &=
            (\SI{2.7166}{\centi\meter\cubed\per\mole})
            [(0.75)^2 - (0.75)^3] \\
            \Aboxed{\Delta \overline V_{\text{mix}} &=
              \SI{0.382}{\centi\meter
            \cubed\per\mole}} \\
            \Delta \overline V_2(X_1) &=
            2(\SI{2.7166}{\centi\meter\cubed\per\mole})
            [(0.75)^3 - (0.75)^2] \\
            \Aboxed{\Delta \overline V_2(X_1) &=
            \SI{0.764}{\centi\meter\cubed\per\mole}} \\
            \Delta \overline V_1 &=
            2(\SI{2.7166}{\centi\meter\cubed\per\mole}
            )(0.75)(1 - (0.75))^2 \\
            \Aboxed{\Delta \overline V_1 &=
            \SI{0.254}{\centi\meter\cubed\per\mole}}
          \end{align*}
        }

      \item GRAPHICALLY (i.e., you will need to plot something)
        verify the partial molar volume change
        of component 1 that you calculated in part (c). You can
        estimate your answer for this part.

        \boxedanswer{
          \centering
          \includegraphics[width=0.6\textwidth]{./assets/fig_3.png}

          \textbf{The intercept is about $V_{\text{mix} =
          \SI{0.76}{\centi\meter\cubed\per\mole}}$.}
        }

    \end{enumerate}

    \pagebreak

  \item Consider a binary phase containing the alloying elements A
    and B, which follows a subregular
    solution model with the following mixing parameters:

    \begin{align*}
      \omega_A &= \SI{10500}{\joule\per\mole} \\
      \omega_B &= \SI{-3000}{\joule\per\mole}
    \end{align*}

    You can assume that both A and B have references states
    consistent with the phase being plotted.

    \begin{enumerate}[(a.)]
      \item On the same plot, plot $T\Delta S_{\text{mix}}$, $\Delta
        H_{\text{mix}}$, and
        $\Delta G_{\text{mix}}$ for this
        phase at \SI{500}{\kelvin} over a compositional range of $X_B
        = 0$ to $X_B = 1$.

        \boxedanswer{
          The subregular solution model has the following form:

          \begin{align*}
            \Delta H_{\text{mix}} &=
            X_1X_2(\alpha_0X_1 + \alpha_1X_2) \tag{8.115} \\
            &
            \begin{aligned}
              X_1 &= X_A \\
              X_2 &= X_B \\
              \alpha_0 &= \omega_A = \SI{10500}{\joule\per\mole} \\
              \alpha_1 &= \omega_B = \SI{-3000}{\joule\per\mole} \\
              X_A &= 1 - X_B \\
            \end{aligned} \\
            \Aboxed{\Delta G_{\text{mix}}^{\text{XS}} &=
              (1 - X_B)X_B[(\SI{10500}{\joule\per\mole})(1 - X_B) -
            (\SI{3000}{\joule\per\mole})X_B]} \\
            \Aboxed{\Delta S_{\text{mix}} &= -R((1 - X_B)\ln (1 -
              X_B) + X_B \ln X_B} \\
              \Aboxed{\Delta G_{\text{mix}} &= \Delta H_{\text{mix}}
              - T\Delta S_{\text{mix}}} \\
            \end{align*}

            \centering
            \includegraphics[width=0.6\textwidth]{./assets/fig_4.png}
          }

          \pagebreak

        \item On a new plot, plot that same three curves at
          \SI{5000}{\kelvin} over
          the same compositional range.
          Describe what you see in this plot and how entropy and
          enthalpy play into the overall Gibbs free
          energy of mixing.

          \boxedanswer{
            At higher temperatures, entropic effects dominate, and
            enthalpic effects diminish.

            \centering
            \includegraphics[width=0.6\textwidth]{./assets/fig_5.png}
          }

          \pagebreak

        \item Using your plot from part (a), graphically determine the
          activity coefficient of B in an alloy
          containing \SI{60}{at\percent} B at \SI{500}{\kelvin}.

          \boxedanswer{
            \begin{align*}
              1
            \end{align*}
            \centering
            \includegraphics[width=0.6\textwidth]{./assets/fig_6.png}

            The intercept at $X_B = 1$ is around $\SI{-3032}{\joule\per\mole}$.

            \begin{align*}
              \Delta G &= RT\ln a \\
              a &= \exp\left(\frac{\Delta G}{RT}\right) \\
              a &=
              \exp\left(\frac{\SI{-3032}{\joule\per\mole}}{(\SI{8.314}{\joule\per\mole\kelvin})(\SI{500}{\kelvin})}\right)
              \\
              \Aboxed{a &= 0.482}
            \end{align*}
          }
      \end{enumerate}

  \end{enumerate}

  \pagebreak

  \section*{Supporting code:}
  \inputminted{julia}{./calculations/src/calculations.jl}

  \end{document}


\begin{document}

\input{./src/titlepage.tex}

\pagebreak

\begin{enumerate}
  \item After careful synthesis, you finally manage to create a new
    substance (call it ‘A’) that maintains a
    liquid phase over a wide range of temperatures and will mix with
    water on warm days but is immiscible
    with water on colder days. The mixture of A and water seems to
    obey the regular solution model to
    an absurd degree. Before publishing your results, you hope to
    characterize binary mixtures of A and
    water in terms of the regular solution model through a series of
    experiments.

    \begin{enumerate}[(a.)]
      \item In preparation, you wish to recall the relationship
        between equilibrium volume fraction and
        the interaction parameter $\chi$. Starting from the mixing free
        energy, derive this relationship.

        \boxedanswer{
          Recall the Gibbs free energy of mixing:

          \begin{equation}
            \Delta G_{\text{mix}} = a_0X_1X_2 + RT(X_1\ln X_1 + X_2
            \ln X_2) \tag{8.110}
          \end{equation}

          Set $X_1 = \phi_A$ and $a_0 = \chi RT$. Note that $X_2 = 1
          - X_1$, and that $\frac{d\Delta G_{\text{mix}}}{dX_1}$

          \begin{align*}
            \frac{d(\Delta G_{\text{mix}}/(RT))}{d\phi_A} &= \chi(1 -
            2\phi_A) + (\ln \phi_A + 1 - \ln (1 - \phi_A) + 1) = 0  \\
            \chi(2\phi_A - 1) &= RT\left(\ln \left[\frac{\phi_A}{1 -
            \phi_A}\right] + 2\right)  \\
            \Aboxed{\chi &= \frac{1}{2\phi_A - 1}\left(\ln
            \left[\frac{\phi_A}{1 - \phi_A}\right]\right)}
          \end{align*}
        }

      \item  Without bothering to measure the initial volume fraction
        of A, you start by cooling the mixture to $T =
        \SI{50}{\degree F}$, observing phase
        separation. Carefully
        drawing a sample from the A-rich phase, you
        then determine that it has a volume fraction of $\phi_A = 0.8$.
        What is the value of $\chi$ that this corresponds
        to?

        \boxedanswer{

          Plug into the above equation:

          \begin{align*}
            \chi &= \frac{1}{2\phi_A - 1}\left(\ln
            \left[\frac{\phi_A}{1 - \phi_A}\right] + 2\right) \\
            \chi &= \frac{1}{2(0.8) - 1}\left(\ln \left[\frac{0.8}{1
            - 0.8}\right]\right) \\
            \Aboxed{\chi &\approx 2.310}
          \end{align*}
        }

      \item In a separate experiment, you prepare a mixture, this
        time taking care to record a volume
        fraction $\phi_A^{(0)} = 0.6$, and place it in your lab
        refrigerator, cooling it to $T = \SI{35}{\degree F}$ . You
        notice an \underline{almost complete separation} of A from water.
        Thinking that you
        might have difficulty measuring the
        possibly trace amounts of water in the A-rich phase, you
        instead note that the A-rich phase occupies
        about \SI{62}{\percent} of the container. Use this
        information to determine
        \si{\percent} the volume fraction of the A-rich phase.

        \boxedanswer{

          We can assume that volume is conserved, and calculate the
          volume fraction as follows:

          \begin{align*}
            \phi_A^{(0)} &= a\phi_A^{(1)} + (1 - a)\phi_A^{(2)} \\
            &
            \begin{aligned}
              \phi_A^{(2)} \approx 0 \\
              a &= 0.62 \\
            \end{aligned} \\
            \phi_A^{(1)} &= \frac{0.6}{0.62} \\
            \Aboxed{\phi_A^{(1)} &= 0.968}
          \end{align*}
        }

      \item The interaction parameter can be expressed as a function
        of temperature as $\chi = \frac{A}{T} + B$,
        where $T$ is measured in Kelvin and $A$ and $B$ are material
        constants. Using the data obtained from your two
        experiments, determine $A$ and $B$.

        \boxedanswer{

          We have two data points, at the first data point, the interaction
          parameter is 5.644 at \SI{50}{\degree F} or \SI{283.15}{\kelvin},
          and the volume fraction is 0.968 at \SI{35}{\degree F} or
          \SI{274.816}{\kelvin}.
          This can be used to find the interaction parameter at this
          temperature,
          and then the linear equation can be fitted.

          \begin{align*}
            \chi_1 &= \frac{A}{T_1} + B \\
            \chi_2 &= \frac{A}{T_2} + B \\
            B &= \chi_1 - \frac{A}{T_1} \\
            A &= \frac{\chi_2 - \chi_1}{\left(\frac{1}{T_2}  -
            \frac{1}{T_1}\right)} \\
            % B &= \chi_1 - \frac{\chi_2 -
            % \chi_1}{\left(\frac{T_1}{T_2}  - 1\right)} \\
            A &= \frac{\chi_2 - \chi_1}{\left(\frac{1}{T_2}  -
            \frac{1}{T_1}\right)} \\
            &
            \begin{aligned}
              \chi_1 &= 2.31 \\
              \chi_2 &= \frac{1}{2\phi_A - 1}\left(\ln
              \left[\frac{\phi_A}{1 - \phi_A}\right]\right) \\
              \chi &= \frac{1}{2(0.968) - 1}\left(\ln
              \left[\frac{(0.968)}{1 - 0.968}\right]\right) \\
              \chi_2 &= 3.66 \\
              T_1 &= \SI{283.15}{\kelvin} \\
              T_2 &= \SI{274.816}{\kelvin} \\
            \end{aligned} \\
            A &= \frac{(3.66) -
            (2.31)}{\left(\frac{1}{(\SI{283.15}{\kelvin})}  -
            \frac{1}{\SI{274.816}{\kelvin}}\right)} \\
            A &= \SI{12375.02}{\kelvin} \\
            B &= (2.31) -
            \frac{(\SI{12375.02}{\kelvin})}{(\SI{283.15}{\kelvin})} \\
            B &= -41.39 \\
            \Aboxed{\chi(T) &= \frac{\SI{12375.02}{\kelvin}}{T} -41.39}
          \end{align*}
        }

      \item Using what you have learned, determine the
        \textit{upper critical
        solution temperature (UCST)} – the
        temperature above which A and water are miscible for all
        compositions.

        \boxedanswer{

          The critical value for $\chi$ is 2. So:

          \begin{align*}
            \chi_c &= \frac{A}{T_c} + B \\
            T_c &= \left[\frac{\chi_c - B}{A}\right]^{-1} \\
            T_c &= \left[\frac{2 +
            41.39}{\SI{12375.02}{\kelvin}}\right]^{-1} \\
            \Aboxed{T_c &= \SI{285.18}{\kelvin}}
          \end{align*}
        }

    \end{enumerate}

    \pagebreak

  \item You have found that the partial molar volume of component 1
    of a two-component system is given by

    \begin{equation}
      \Delta \overline V_1 = 2aX_1X_2^2
    \end{equation}

    where $a = \SI{2.7166}{\centi\meter\cubed\per\mole}$

    \begin{enumerate}[(a.)]
      \item Determine an equation for the change in partial molar
        volume of component 2 during mixing
        as a function of composition.

        \boxedanswer{
          Use the Gibbs-Duhem relation:

          \begin{align*}
            X_1 d\Delta \overline{V}_1 + X_2 d\Delta
            \overline V_2 &= 0 \\
            d\Delta \overline V_2 &= -\frac{X_1}{X_2}d\Delta
            \overline{V}_1 \\
            &
            \begin{aligned}
              X_2 &= 1 - X_1 \\
              \Delta \overline V_1 &= 2 a X_1 (1 - X_1)^2 \\
              \Delta \overline V_1 &= 2 a (X_1  - 2X_1^2 + X_1^3) \\
              d\Delta \overline V_1 &= 2 a (3X_1^2 - 4X_1 + 1)dX_1 \\
              d\Delta \overline V_1 &= 2 a (3X_1 - 1)(X_1 - 1)dX_1 \\
            \end{aligned} \\
            d\Delta \overline V_2 &= -\frac{X_1}{\cancel{1-  X_1}}2 a
            (3X_1 - 1)(\cancel{X_1 - 1})dX_1 \\
            d\Delta \overline V_2 &= 2aX_1(3X_1 - 1)dX_1 \\
            \int d\Delta \overline V_2 &= \int 2aX_1(3X_1 - 1)dX_1 \\
            \Delta \overline V_2(X_1) &= 2a\left[X_1^3 -
            \frac{X_1^2}{2}\right] + C \\
            &
            \begin{aligned}
              \Delta \overline V_2(0) &= 0 \\
              C &= 0 \\
            \end{aligned} \\
            \Aboxed{\Delta \overline V_2(X_1) &= 2aX_1^3 - aX_1^2}
          \end{align*}
        }

      \item Determine an equation for the total volume change during
        mixing as a function of composition.

        \boxedanswer{
          The total volume change can be written:

          \begin{align*}
            \Delta \overline V_{\text{mix}} &= X_1 \Delta \overline
            V_1 + X_2 \Delta \overline V_2 \\
            &
            \begin{aligned}
              \Delta \overline V_1 &= 2aX_1 (1 - X_1)^2 \\
              \Delta \overline V_2(X_1) &= 2aX_1^3 - aX_1^2 \\
              X_2 &= 1 - X_1 \\
            \end{aligned} \\
            \Delta \overline V_{\text{mix}} &= X_12aX_1 (1 - X_1)^2 +
            (1 - X_1)(2aX_1^3 - aX_1^2) \\
            \Delta \overline V_{\text{mix}} &= 2aX_1^2 - 4aX_1^3 +
            \cancel{2aX_1^4} + 2aX_1^3 - aX_1^2 -
            \cancel{2aX_1^4} + aX_1^3 \\
            \Aboxed{\Delta \overline V_{\text{mix}} &= aX_1^2 - aX_1^3}
          \end{align*}
        }

      \item Calculate the total volume change of the mixture, the
        partial molar volume change of component 1, and the partial
        molar volume change of component 2
        (three separate numerical answers) when
        \SI{75}{\mole} of component 1 are mixed with \SI{25}{\mole}
        of component 2.

        \boxedanswer{
          \begin{align*}
            \Delta \overline V_{\text{mix}} &= aX_1^2 - aX_1^3 \\
            &
            \begin{aligned}
              X_1 &= \frac{75}{75 + 25} = 0.75 \\
              a &= \SI{2.7166}{\centi\meter\cubed\per\mole}
            \end{aligned} \\
            \Delta \overline V_{\text{mix}} &=
            (\SI{2.7166}{\centi\meter\cubed\per\mole})
            [(0.75)^2 - (0.75)^3] \\
            \Aboxed{\Delta \overline V_{\text{mix}} &=
              \SI{0.382}{\centi\meter
            \cubed\per\mole}} \\
            \Delta \overline V_2(X_1) &=
            2(\SI{2.7166}{\centi\meter\cubed\per\mole})
            [(0.75)^3 - (0.75)^2] \\
            \Aboxed{\Delta \overline V_2(X_1) &=
            \SI{0.764}{\centi\meter\cubed\per\mole}} \\
            \Delta \overline V_1 &=
            2(\SI{2.7166}{\centi\meter\cubed\per\mole}
            )(0.75)(1 - (0.75))^2 \\
            \Aboxed{\Delta \overline V_1 &=
            \SI{0.254}{\centi\meter\cubed\per\mole}}
          \end{align*}
        }

      \item GRAPHICALLY (i.e., you will need to plot something)
        verify the partial molar volume change
        of component 1 that you calculated in part (c). You can
        estimate your answer for this part.

        \boxedanswer{
          \centering
          \includegraphics[width=0.6\textwidth]{./assets/fig_3.png}

          \textbf{The intercept is about $V_{\text{mix} =
          \SI{0.76}{\centi\meter\cubed\per\mole}}$.}
        }

    \end{enumerate}

    \pagebreak

  \item Consider a binary phase containing the alloying elements A
    and B, which follows a subregular
    solution model with the following mixing parameters:

    \begin{align*}
      \omega_A &= \SI{10500}{\joule\per\mole} \\
      \omega_B &= \SI{-3000}{\joule\per\mole}
    \end{align*}

    You can assume that both A and B have references states
    consistent with the phase being plotted.

    \begin{enumerate}[(a.)]
      \item On the same plot, plot $T\Delta S_{\text{mix}}$, $\Delta
        H_{\text{mix}}$, and
        $\Delta G_{\text{mix}}$ for this
        phase at \SI{500}{\kelvin} over a compositional range of $X_B
        = 0$ to $X_B = 1$.

        \boxedanswer{
          The subregular solution model has the following form:

          \begin{align*}
            \Delta H_{\text{mix}} &=
            X_1X_2(\alpha_0X_1 + \alpha_1X_2) \tag{8.115} \\
            &
            \begin{aligned}
              X_1 &= X_A \\
              X_2 &= X_B \\
              \alpha_0 &= \omega_A = \SI{10500}{\joule\per\mole} \\
              \alpha_1 &= \omega_B = \SI{-3000}{\joule\per\mole} \\
              X_A &= 1 - X_B \\
            \end{aligned} \\
            \Aboxed{\Delta G_{\text{mix}}^{\text{XS}} &=
              (1 - X_B)X_B[(\SI{10500}{\joule\per\mole})(1 - X_B) -
            (\SI{3000}{\joule\per\mole})X_B]} \\
            \Aboxed{\Delta S_{\text{mix}} &= -R((1 - X_B)\ln (1 -
              X_B) + X_B \ln X_B} \\
              \Aboxed{\Delta G_{\text{mix}} &= \Delta H_{\text{mix}}
              - T\Delta S_{\text{mix}}} \\
            \end{align*}

            \centering
            \includegraphics[width=0.6\textwidth]{./assets/fig_4.png}
          }

          \pagebreak

        \item On a new plot, plot that same three curves at
          \SI{5000}{\kelvin} over
          the same compositional range.
          Describe what you see in this plot and how entropy and
          enthalpy play into the overall Gibbs free
          energy of mixing.

          \boxedanswer{
            At higher temperatures, entropic effects dominate, and
            enthalpic effects diminish.

            \centering
            \includegraphics[width=0.6\textwidth]{./assets/fig_5.png}
          }

          \pagebreak

        \item Using your plot from part (a), graphically determine the
          activity coefficient of B in an alloy
          containing \SI{60}{at\percent} B at \SI{500}{\kelvin}.

          \boxedanswer{
            \begin{align*}
              1
            \end{align*}
            \centering
            \includegraphics[width=0.6\textwidth]{./assets/fig_6.png}

            The intercept at $X_B = 1$ is around $\SI{-3032}{\joule\per\mole}$.

            \begin{align*}
              \Delta G &= RT\ln a \\
              a &= \exp\left(\frac{\Delta G}{RT}\right) \\
              a &=
              \exp\left(\frac{\SI{-3032}{\joule\per\mole}}{(\SI{8.314}{\joule\per\mole\kelvin})(\SI{500}{\kelvin})}\right)
              \\
              \Aboxed{a &= 0.482}
            \end{align*}
          }
      \end{enumerate}

  \end{enumerate}

  \pagebreak

  \section*{Supporting code:}
  \inputminted{julia}{./calculations/src/calculations.jl}

  \end{document}


\begin{document}

\input{./src/titlepage.tex}

\pagebreak

\begin{enumerate}
  \item After careful synthesis, you finally manage to create a new
    substance (call it ‘A’) that maintains a
    liquid phase over a wide range of temperatures and will mix with
    water on warm days but is immiscible
    with water on colder days. The mixture of A and water seems to
    obey the regular solution model to
    an absurd degree. Before publishing your results, you hope to
    characterize binary mixtures of A and
    water in terms of the regular solution model through a series of
    experiments.

    \begin{enumerate}[(a.)]
      \item In preparation, you wish to recall the relationship
        between equilibrium volume fraction and
        the interaction parameter $\chi$. Starting from the mixing free
        energy, derive this relationship.

        \boxedanswer{
          Recall the Gibbs free energy of mixing:

          \begin{equation}
            \Delta G_{\text{mix}} = a_0X_1X_2 + RT(X_1\ln X_1 + X_2
            \ln X_2) \tag{8.110}
          \end{equation}

          Set $X_1 = \phi_A$ and $a_0 = \chi RT$. Note that $X_2 = 1
          - X_1$, and that $\frac{d\Delta G_{\text{mix}}}{dX_1}$

          \begin{align*}
            \frac{d(\Delta G_{\text{mix}}/(RT))}{d\phi_A} &= \chi(1 -
            2\phi_A) + (\ln \phi_A + 1 - \ln (1 - \phi_A) + 1) = 0  \\
            \chi(2\phi_A - 1) &= RT\left(\ln \left[\frac{\phi_A}{1 -
            \phi_A}\right] + 2\right)  \\
            \Aboxed{\chi &= \frac{1}{2\phi_A - 1}\left(\ln
            \left[\frac{\phi_A}{1 - \phi_A}\right]\right)}
          \end{align*}
        }

      \item  Without bothering to measure the initial volume fraction
        of A, you start by cooling the mixture to $T =
        \SI{50}{\degree F}$, observing phase
        separation. Carefully
        drawing a sample from the A-rich phase, you
        then determine that it has a volume fraction of $\phi_A = 0.8$.
        What is the value of $\chi$ that this corresponds
        to?

        \boxedanswer{

          Plug into the above equation:

          \begin{align*}
            \chi &= \frac{1}{2\phi_A - 1}\left(\ln
            \left[\frac{\phi_A}{1 - \phi_A}\right] + 2\right) \\
            \chi &= \frac{1}{2(0.8) - 1}\left(\ln \left[\frac{0.8}{1
            - 0.8}\right]\right) \\
            \Aboxed{\chi &\approx 2.310}
          \end{align*}
        }

      \item In a separate experiment, you prepare a mixture, this
        time taking care to record a volume
        fraction $\phi_A^{(0)} = 0.6$, and place it in your lab
        refrigerator, cooling it to $T = \SI{35}{\degree F}$ . You
        notice an \underline{almost complete separation} of A from water.
        Thinking that you
        might have difficulty measuring the
        possibly trace amounts of water in the A-rich phase, you
        instead note that the A-rich phase occupies
        about \SI{62}{\percent} of the container. Use this
        information to determine
        \si{\percent} the volume fraction of the A-rich phase.

        \boxedanswer{

          We can assume that volume is conserved, and calculate the
          volume fraction as follows:

          \begin{align*}
            \phi_A^{(0)} &= a\phi_A^{(1)} + (1 - a)\phi_A^{(2)} \\
            &
            \begin{aligned}
              \phi_A^{(2)} \approx 0 \\
              a &= 0.62 \\
            \end{aligned} \\
            \phi_A^{(1)} &= \frac{0.6}{0.62} \\
            \Aboxed{\phi_A^{(1)} &= 0.968}
          \end{align*}
        }

      \item The interaction parameter can be expressed as a function
        of temperature as $\chi = \frac{A}{T} + B$,
        where $T$ is measured in Kelvin and $A$ and $B$ are material
        constants. Using the data obtained from your two
        experiments, determine $A$ and $B$.

        \boxedanswer{

          We have two data points, at the first data point, the interaction
          parameter is 5.644 at \SI{50}{\degree F} or \SI{283.15}{\kelvin},
          and the volume fraction is 0.968 at \SI{35}{\degree F} or
          \SI{274.816}{\kelvin}.
          This can be used to find the interaction parameter at this
          temperature,
          and then the linear equation can be fitted.

          \begin{align*}
            \chi_1 &= \frac{A}{T_1} + B \\
            \chi_2 &= \frac{A}{T_2} + B \\
            B &= \chi_1 - \frac{A}{T_1} \\
            A &= \frac{\chi_2 - \chi_1}{\left(\frac{1}{T_2}  -
            \frac{1}{T_1}\right)} \\
            % B &= \chi_1 - \frac{\chi_2 -
            % \chi_1}{\left(\frac{T_1}{T_2}  - 1\right)} \\
            A &= \frac{\chi_2 - \chi_1}{\left(\frac{1}{T_2}  -
            \frac{1}{T_1}\right)} \\
            &
            \begin{aligned}
              \chi_1 &= 2.31 \\
              \chi_2 &= \frac{1}{2\phi_A - 1}\left(\ln
              \left[\frac{\phi_A}{1 - \phi_A}\right]\right) \\
              \chi &= \frac{1}{2(0.968) - 1}\left(\ln
              \left[\frac{(0.968)}{1 - 0.968}\right]\right) \\
              \chi_2 &= 3.66 \\
              T_1 &= \SI{283.15}{\kelvin} \\
              T_2 &= \SI{274.816}{\kelvin} \\
            \end{aligned} \\
            A &= \frac{(3.66) -
            (2.31)}{\left(\frac{1}{(\SI{283.15}{\kelvin})}  -
            \frac{1}{\SI{274.816}{\kelvin}}\right)} \\
            A &= \SI{12375.02}{\kelvin} \\
            B &= (2.31) -
            \frac{(\SI{12375.02}{\kelvin})}{(\SI{283.15}{\kelvin})} \\
            B &= -41.39 \\
            \Aboxed{\chi(T) &= \frac{\SI{12375.02}{\kelvin}}{T} -41.39}
          \end{align*}
        }

      \item Using what you have learned, determine the
        \textit{upper critical
        solution temperature (UCST)} – the
        temperature above which A and water are miscible for all
        compositions.

        \boxedanswer{

          The critical value for $\chi$ is 2. So:

          \begin{align*}
            \chi_c &= \frac{A}{T_c} + B \\
            T_c &= \left[\frac{\chi_c - B}{A}\right]^{-1} \\
            T_c &= \left[\frac{2 +
            41.39}{\SI{12375.02}{\kelvin}}\right]^{-1} \\
            \Aboxed{T_c &= \SI{285.18}{\kelvin}}
          \end{align*}
        }

    \end{enumerate}

    \pagebreak

  \item You have found that the partial molar volume of component 1
    of a two-component system is given by

    \begin{equation}
      \Delta \overline V_1 = 2aX_1X_2^2
    \end{equation}

    where $a = \SI{2.7166}{\centi\meter\cubed\per\mole}$

    \begin{enumerate}[(a.)]
      \item Determine an equation for the change in partial molar
        volume of component 2 during mixing
        as a function of composition.

        \boxedanswer{
          Use the Gibbs-Duhem relation:

          \begin{align*}
            X_1 d\Delta \overline{V}_1 + X_2 d\Delta
            \overline V_2 &= 0 \\
            d\Delta \overline V_2 &= -\frac{X_1}{X_2}d\Delta
            \overline{V}_1 \\
            &
            \begin{aligned}
              X_2 &= 1 - X_1 \\
              \Delta \overline V_1 &= 2 a X_1 (1 - X_1)^2 \\
              \Delta \overline V_1 &= 2 a (X_1  - 2X_1^2 + X_1^3) \\
              d\Delta \overline V_1 &= 2 a (3X_1^2 - 4X_1 + 1)dX_1 \\
              d\Delta \overline V_1 &= 2 a (3X_1 - 1)(X_1 - 1)dX_1 \\
            \end{aligned} \\
            d\Delta \overline V_2 &= -\frac{X_1}{\cancel{1-  X_1}}2 a
            (3X_1 - 1)(\cancel{X_1 - 1})dX_1 \\
            d\Delta \overline V_2 &= 2aX_1(3X_1 - 1)dX_1 \\
            \int d\Delta \overline V_2 &= \int 2aX_1(3X_1 - 1)dX_1 \\
            \Delta \overline V_2(X_1) &= 2a\left[X_1^3 -
            \frac{X_1^2}{2}\right] + C \\
            &
            \begin{aligned}
              \Delta \overline V_2(0) &= 0 \\
              C &= 0 \\
            \end{aligned} \\
            \Aboxed{\Delta \overline V_2(X_1) &= 2aX_1^3 - aX_1^2}
          \end{align*}
        }

      \item Determine an equation for the total volume change during
        mixing as a function of composition.

        \boxedanswer{
          The total volume change can be written:

          \begin{align*}
            \Delta \overline V_{\text{mix}} &= X_1 \Delta \overline
            V_1 + X_2 \Delta \overline V_2 \\
            &
            \begin{aligned}
              \Delta \overline V_1 &= 2aX_1 (1 - X_1)^2 \\
              \Delta \overline V_2(X_1) &= 2aX_1^3 - aX_1^2 \\
              X_2 &= 1 - X_1 \\
            \end{aligned} \\
            \Delta \overline V_{\text{mix}} &= X_12aX_1 (1 - X_1)^2 +
            (1 - X_1)(2aX_1^3 - aX_1^2) \\
            \Delta \overline V_{\text{mix}} &= 2aX_1^2 - 4aX_1^3 +
            \cancel{2aX_1^4} + 2aX_1^3 - aX_1^2 -
            \cancel{2aX_1^4} + aX_1^3 \\
            \Aboxed{\Delta \overline V_{\text{mix}} &= aX_1^2 - aX_1^3}
          \end{align*}
        }

      \item Calculate the total volume change of the mixture, the
        partial molar volume change of component 1, and the partial
        molar volume change of component 2
        (three separate numerical answers) when
        \SI{75}{\mole} of component 1 are mixed with \SI{25}{\mole}
        of component 2.

        \boxedanswer{
          \begin{align*}
            \Delta \overline V_{\text{mix}} &= aX_1^2 - aX_1^3 \\
            &
            \begin{aligned}
              X_1 &= \frac{75}{75 + 25} = 0.75 \\
              a &= \SI{2.7166}{\centi\meter\cubed\per\mole}
            \end{aligned} \\
            \Delta \overline V_{\text{mix}} &=
            (\SI{2.7166}{\centi\meter\cubed\per\mole})
            [(0.75)^2 - (0.75)^3] \\
            \Aboxed{\Delta \overline V_{\text{mix}} &=
              \SI{0.382}{\centi\meter
            \cubed\per\mole}} \\
            \Delta \overline V_2(X_1) &=
            2(\SI{2.7166}{\centi\meter\cubed\per\mole})
            [(0.75)^3 - (0.75)^2] \\
            \Aboxed{\Delta \overline V_2(X_1) &=
            \SI{0.764}{\centi\meter\cubed\per\mole}} \\
            \Delta \overline V_1 &=
            2(\SI{2.7166}{\centi\meter\cubed\per\mole}
            )(0.75)(1 - (0.75))^2 \\
            \Aboxed{\Delta \overline V_1 &=
            \SI{0.254}{\centi\meter\cubed\per\mole}}
          \end{align*}
        }

      \item GRAPHICALLY (i.e., you will need to plot something)
        verify the partial molar volume change
        of component 1 that you calculated in part (c). You can
        estimate your answer for this part.

        \boxedanswer{
          \centering
          \includegraphics[width=0.6\textwidth]{./assets/fig_3.png}

          \textbf{The intercept is about $V_{\text{mix} =
          \SI{0.76}{\centi\meter\cubed\per\mole}}$.}
        }

    \end{enumerate}

    \pagebreak

  \item Consider a binary phase containing the alloying elements A
    and B, which follows a subregular
    solution model with the following mixing parameters:

    \begin{align*}
      \omega_A &= \SI{10500}{\joule\per\mole} \\
      \omega_B &= \SI{-3000}{\joule\per\mole}
    \end{align*}

    You can assume that both A and B have references states
    consistent with the phase being plotted.

    \begin{enumerate}[(a.)]
      \item On the same plot, plot $T\Delta S_{\text{mix}}$, $\Delta
        H_{\text{mix}}$, and
        $\Delta G_{\text{mix}}$ for this
        phase at \SI{500}{\kelvin} over a compositional range of $X_B
        = 0$ to $X_B = 1$.

        \boxedanswer{
          The subregular solution model has the following form:

          \begin{align*}
            \Delta H_{\text{mix}} &=
            X_1X_2(\alpha_0X_1 + \alpha_1X_2) \tag{8.115} \\
            &
            \begin{aligned}
              X_1 &= X_A \\
              X_2 &= X_B \\
              \alpha_0 &= \omega_A = \SI{10500}{\joule\per\mole} \\
              \alpha_1 &= \omega_B = \SI{-3000}{\joule\per\mole} \\
              X_A &= 1 - X_B \\
            \end{aligned} \\
            \Aboxed{\Delta G_{\text{mix}}^{\text{XS}} &=
              (1 - X_B)X_B[(\SI{10500}{\joule\per\mole})(1 - X_B) -
            (\SI{3000}{\joule\per\mole})X_B]} \\
            \Aboxed{\Delta S_{\text{mix}} &= -R((1 - X_B)\ln (1 -
              X_B) + X_B \ln X_B} \\
              \Aboxed{\Delta G_{\text{mix}} &= \Delta H_{\text{mix}}
              - T\Delta S_{\text{mix}}} \\
            \end{align*}

            \centering
            \includegraphics[width=0.6\textwidth]{./assets/fig_4.png}
          }

          \pagebreak

        \item On a new plot, plot that same three curves at
          \SI{5000}{\kelvin} over
          the same compositional range.
          Describe what you see in this plot and how entropy and
          enthalpy play into the overall Gibbs free
          energy of mixing.

          \boxedanswer{
            At higher temperatures, entropic effects dominate, and
            enthalpic effects diminish.

            \centering
            \includegraphics[width=0.6\textwidth]{./assets/fig_5.png}
          }

          \pagebreak

        \item Using your plot from part (a), graphically determine the
          activity coefficient of B in an alloy
          containing \SI{60}{at\percent} B at \SI{500}{\kelvin}.

          \boxedanswer{
            \begin{align*}
              1
            \end{align*}
            \centering
            \includegraphics[width=0.6\textwidth]{./assets/fig_6.png}

            The intercept at $X_B = 1$ is around $\SI{-3032}{\joule\per\mole}$.

            \begin{align*}
              \Delta G &= RT\ln a \\
              a &= \exp\left(\frac{\Delta G}{RT}\right) \\
              a &=
              \exp\left(\frac{\SI{-3032}{\joule\per\mole}}{(\SI{8.314}{\joule\per\mole\kelvin})(\SI{500}{\kelvin})}\right)
              \\
              \Aboxed{a &= 0.482}
            \end{align*}
          }
      \end{enumerate}

  \end{enumerate}

  \pagebreak

  \section*{Supporting code:}
  \inputminted{julia}{./calculations/src/calculations.jl}

  \end{document}


\begin{document}

\input{./src/titlepage.tex}

\pagebreak

\begin{enumerate}
  \item After careful synthesis, you finally manage to create a new
    substance (call it ‘A’) that maintains a
    liquid phase over a wide range of temperatures and will mix with
    water on warm days but is immiscible
    with water on colder days. The mixture of A and water seems to
    obey the regular solution model to
    an absurd degree. Before publishing your results, you hope to
    characterize binary mixtures of A and
    water in terms of the regular solution model through a series of
    experiments.

    \begin{enumerate}[(a.)]
      \item In preparation, you wish to recall the relationship
        between equilibrium volume fraction and
        the interaction parameter $\chi$. Starting from the mixing free
        energy, derive this relationship.

        \boxedanswer{
          Recall the Gibbs free energy of mixing:

          \begin{equation}
            \Delta G_{\text{mix}} = a_0X_1X_2 + RT(X_1\ln X_1 + X_2
            \ln X_2) \tag{8.110}
          \end{equation}

          Set $X_1 = \phi_A$ and $a_0 = \chi RT$. Note that $X_2 = 1
          - X_1$, and that $\frac{d\Delta G_{\text{mix}}}{dX_1}$

          \begin{align*}
            \frac{d(\Delta G_{\text{mix}}/(RT))}{d\phi_A} &= \chi(1 -
            2\phi_A) + (\ln \phi_A + 1 - \ln (1 - \phi_A) + 1) = 0  \\
            \chi(2\phi_A - 1) &= RT\left(\ln \left[\frac{\phi_A}{1 -
            \phi_A}\right] + 2\right)  \\
            \Aboxed{\chi &= \frac{1}{2\phi_A - 1}\left(\ln
            \left[\frac{\phi_A}{1 - \phi_A}\right]\right)}
          \end{align*}
        }

      \item  Without bothering to measure the initial volume fraction
        of A, you start by cooling the mixture to $T =
        \SI{50}{\degree F}$, observing phase
        separation. Carefully
        drawing a sample from the A-rich phase, you
        then determine that it has a volume fraction of $\phi_A = 0.8$.
        What is the value of $\chi$ that this corresponds
        to?

        \boxedanswer{

          Plug into the above equation:

          \begin{align*}
            \chi &= \frac{1}{2\phi_A - 1}\left(\ln
            \left[\frac{\phi_A}{1 - \phi_A}\right] + 2\right) \\
            \chi &= \frac{1}{2(0.8) - 1}\left(\ln \left[\frac{0.8}{1
            - 0.8}\right]\right) \\
            \Aboxed{\chi &\approx 2.310}
          \end{align*}
        }

      \item In a separate experiment, you prepare a mixture, this
        time taking care to record a volume
        fraction $\phi_A^{(0)} = 0.6$, and place it in your lab
        refrigerator, cooling it to $T = \SI{35}{\degree F}$ . You
        notice an \underline{almost complete separation} of A from water.
        Thinking that you
        might have difficulty measuring the
        possibly trace amounts of water in the A-rich phase, you
        instead note that the A-rich phase occupies
        about \SI{62}{\percent} of the container. Use this
        information to determine
        \si{\percent} the volume fraction of the A-rich phase.

        \boxedanswer{

          We can assume that volume is conserved, and calculate the
          volume fraction as follows:

          \begin{align*}
            \phi_A^{(0)} &= a\phi_A^{(1)} + (1 - a)\phi_A^{(2)} \\
            &
            \begin{aligned}
              \phi_A^{(2)} \approx 0 \\
              a &= 0.62 \\
            \end{aligned} \\
            \phi_A^{(1)} &= \frac{0.6}{0.62} \\
            \Aboxed{\phi_A^{(1)} &= 0.968}
          \end{align*}
        }

      \item The interaction parameter can be expressed as a function
        of temperature as $\chi = \frac{A}{T} + B$,
        where $T$ is measured in Kelvin and $A$ and $B$ are material
        constants. Using the data obtained from your two
        experiments, determine $A$ and $B$.

        \boxedanswer{

          We have two data points, at the first data point, the interaction
          parameter is 5.644 at \SI{50}{\degree F} or \SI{283.15}{\kelvin},
          and the volume fraction is 0.968 at \SI{35}{\degree F} or
          \SI{274.816}{\kelvin}.
          This can be used to find the interaction parameter at this
          temperature,
          and then the linear equation can be fitted.

          \begin{align*}
            \chi_1 &= \frac{A}{T_1} + B \\
            \chi_2 &= \frac{A}{T_2} + B \\
            B &= \chi_1 - \frac{A}{T_1} \\
            A &= \frac{\chi_2 - \chi_1}{\left(\frac{1}{T_2}  -
            \frac{1}{T_1}\right)} \\
            % B &= \chi_1 - \frac{\chi_2 -
            % \chi_1}{\left(\frac{T_1}{T_2}  - 1\right)} \\
            A &= \frac{\chi_2 - \chi_1}{\left(\frac{1}{T_2}  -
            \frac{1}{T_1}\right)} \\
            &
            \begin{aligned}
              \chi_1 &= 2.31 \\
              \chi_2 &= \frac{1}{2\phi_A - 1}\left(\ln
              \left[\frac{\phi_A}{1 - \phi_A}\right]\right) \\
              \chi &= \frac{1}{2(0.968) - 1}\left(\ln
              \left[\frac{(0.968)}{1 - 0.968}\right]\right) \\
              \chi_2 &= 3.66 \\
              T_1 &= \SI{283.15}{\kelvin} \\
              T_2 &= \SI{274.816}{\kelvin} \\
            \end{aligned} \\
            A &= \frac{(3.66) -
            (2.31)}{\left(\frac{1}{(\SI{283.15}{\kelvin})}  -
            \frac{1}{\SI{274.816}{\kelvin}}\right)} \\
            A &= \SI{12375.02}{\kelvin} \\
            B &= (2.31) -
            \frac{(\SI{12375.02}{\kelvin})}{(\SI{283.15}{\kelvin})} \\
            B &= -41.39 \\
            \Aboxed{\chi(T) &= \frac{\SI{12375.02}{\kelvin}}{T} -41.39}
          \end{align*}
        }

      \item Using what you have learned, determine the
        \textit{upper critical
        solution temperature (UCST)} – the
        temperature above which A and water are miscible for all
        compositions.

        \boxedanswer{

          The critical value for $\chi$ is 2. So:

          \begin{align*}
            \chi_c &= \frac{A}{T_c} + B \\
            T_c &= \left[\frac{\chi_c - B}{A}\right]^{-1} \\
            T_c &= \left[\frac{2 +
            41.39}{\SI{12375.02}{\kelvin}}\right]^{-1} \\
            \Aboxed{T_c &= \SI{285.18}{\kelvin}}
          \end{align*}
        }

    \end{enumerate}

    \pagebreak

  \item You have found that the partial molar volume of component 1
    of a two-component system is given by

    \begin{equation}
      \Delta \overline V_1 = 2aX_1X_2^2
    \end{equation}

    where $a = \SI{2.7166}{\centi\meter\cubed\per\mole}$

    \begin{enumerate}[(a.)]
      \item Determine an equation for the change in partial molar
        volume of component 2 during mixing
        as a function of composition.

        \boxedanswer{
          Use the Gibbs-Duhem relation:

          \begin{align*}
            X_1 d\Delta \overline{V}_1 + X_2 d\Delta
            \overline V_2 &= 0 \\
            d\Delta \overline V_2 &= -\frac{X_1}{X_2}d\Delta
            \overline{V}_1 \\
            &
            \begin{aligned}
              X_2 &= 1 - X_1 \\
              \Delta \overline V_1 &= 2 a X_1 (1 - X_1)^2 \\
              \Delta \overline V_1 &= 2 a (X_1  - 2X_1^2 + X_1^3) \\
              d\Delta \overline V_1 &= 2 a (3X_1^2 - 4X_1 + 1)dX_1 \\
              d\Delta \overline V_1 &= 2 a (3X_1 - 1)(X_1 - 1)dX_1 \\
            \end{aligned} \\
            d\Delta \overline V_2 &= -\frac{X_1}{\cancel{1-  X_1}}2 a
            (3X_1 - 1)(\cancel{X_1 - 1})dX_1 \\
            d\Delta \overline V_2 &= 2aX_1(3X_1 - 1)dX_1 \\
            \int d\Delta \overline V_2 &= \int 2aX_1(3X_1 - 1)dX_1 \\
            \Delta \overline V_2(X_1) &= 2a\left[X_1^3 -
            \frac{X_1^2}{2}\right] + C \\
            &
            \begin{aligned}
              \Delta \overline V_2(0) &= 0 \\
              C &= 0 \\
            \end{aligned} \\
            \Aboxed{\Delta \overline V_2(X_1) &= 2aX_1^3 - aX_1^2}
          \end{align*}
        }

      \item Determine an equation for the total volume change during
        mixing as a function of composition.

        \boxedanswer{
          The total volume change can be written:

          \begin{align*}
            \Delta \overline V_{\text{mix}} &= X_1 \Delta \overline
            V_1 + X_2 \Delta \overline V_2 \\
            &
            \begin{aligned}
              \Delta \overline V_1 &= 2aX_1 (1 - X_1)^2 \\
              \Delta \overline V_2(X_1) &= 2aX_1^3 - aX_1^2 \\
              X_2 &= 1 - X_1 \\
            \end{aligned} \\
            \Delta \overline V_{\text{mix}} &= X_12aX_1 (1 - X_1)^2 +
            (1 - X_1)(2aX_1^3 - aX_1^2) \\
            \Delta \overline V_{\text{mix}} &= 2aX_1^2 - 4aX_1^3 +
            \cancel{2aX_1^4} + 2aX_1^3 - aX_1^2 -
            \cancel{2aX_1^4} + aX_1^3 \\
            \Aboxed{\Delta \overline V_{\text{mix}} &= aX_1^2 - aX_1^3}
          \end{align*}
        }

      \item Calculate the total volume change of the mixture, the
        partial molar volume change of component 1, and the partial
        molar volume change of component 2
        (three separate numerical answers) when
        \SI{75}{\mole} of component 1 are mixed with \SI{25}{\mole}
        of component 2.

        \boxedanswer{
          \begin{align*}
            \Delta \overline V_{\text{mix}} &= aX_1^2 - aX_1^3 \\
            &
            \begin{aligned}
              X_1 &= \frac{75}{75 + 25} = 0.75 \\
              a &= \SI{2.7166}{\centi\meter\cubed\per\mole}
            \end{aligned} \\
            \Delta \overline V_{\text{mix}} &=
            (\SI{2.7166}{\centi\meter\cubed\per\mole})
            [(0.75)^2 - (0.75)^3] \\
            \Aboxed{\Delta \overline V_{\text{mix}} &=
              \SI{0.382}{\centi\meter
            \cubed\per\mole}} \\
            \Delta \overline V_2(X_1) &=
            2(\SI{2.7166}{\centi\meter\cubed\per\mole})
            [(0.75)^3 - (0.75)^2] \\
            \Aboxed{\Delta \overline V_2(X_1) &=
            \SI{0.764}{\centi\meter\cubed\per\mole}} \\
            \Delta \overline V_1 &=
            2(\SI{2.7166}{\centi\meter\cubed\per\mole}
            )(0.75)(1 - (0.75))^2 \\
            \Aboxed{\Delta \overline V_1 &=
            \SI{0.254}{\centi\meter\cubed\per\mole}}
          \end{align*}
        }

      \item GRAPHICALLY (i.e., you will need to plot something)
        verify the partial molar volume change
        of component 1 that you calculated in part (c). You can
        estimate your answer for this part.

        \boxedanswer{
          \centering
          \includegraphics[width=0.6\textwidth]{./assets/fig_3.png}

          \textbf{The intercept is about $V_{\text{mix} =
          \SI{0.76}{\centi\meter\cubed\per\mole}}$.}
        }

    \end{enumerate}

    \pagebreak

  \item Consider a binary phase containing the alloying elements A
    and B, which follows a subregular
    solution model with the following mixing parameters:

    \begin{align*}
      \omega_A &= \SI{10500}{\joule\per\mole} \\
      \omega_B &= \SI{-3000}{\joule\per\mole}
    \end{align*}

    You can assume that both A and B have references states
    consistent with the phase being plotted.

    \begin{enumerate}[(a.)]
      \item On the same plot, plot $T\Delta S_{\text{mix}}$, $\Delta
        H_{\text{mix}}$, and
        $\Delta G_{\text{mix}}$ for this
        phase at \SI{500}{\kelvin} over a compositional range of $X_B
        = 0$ to $X_B = 1$.

        \boxedanswer{
          The subregular solution model has the following form:

          \begin{align*}
            \Delta H_{\text{mix}} &=
            X_1X_2(\alpha_0X_1 + \alpha_1X_2) \tag{8.115} \\
            &
            \begin{aligned}
              X_1 &= X_A \\
              X_2 &= X_B \\
              \alpha_0 &= \omega_A = \SI{10500}{\joule\per\mole} \\
              \alpha_1 &= \omega_B = \SI{-3000}{\joule\per\mole} \\
              X_A &= 1 - X_B \\
            \end{aligned} \\
            \Aboxed{\Delta G_{\text{mix}}^{\text{XS}} &=
              (1 - X_B)X_B[(\SI{10500}{\joule\per\mole})(1 - X_B) -
            (\SI{3000}{\joule\per\mole})X_B]} \\
            \Aboxed{\Delta S_{\text{mix}} &= -R((1 - X_B)\ln (1 -
              X_B) + X_B \ln X_B} \\
              \Aboxed{\Delta G_{\text{mix}} &= \Delta H_{\text{mix}}
              - T\Delta S_{\text{mix}}} \\
            \end{align*}

            \centering
            \includegraphics[width=0.6\textwidth]{./assets/fig_4.png}
          }

          \pagebreak

        \item On a new plot, plot that same three curves at
          \SI{5000}{\kelvin} over
          the same compositional range.
          Describe what you see in this plot and how entropy and
          enthalpy play into the overall Gibbs free
          energy of mixing.

          \boxedanswer{
            At higher temperatures, entropic effects dominate, and
            enthalpic effects diminish.

            \centering
            \includegraphics[width=0.6\textwidth]{./assets/fig_5.png}
          }

          \pagebreak

        \item Using your plot from part (a), graphically determine the
          activity coefficient of B in an alloy
          containing \SI{60}{at\percent} B at \SI{500}{\kelvin}.

          \boxedanswer{
            \begin{align*}
              1
            \end{align*}
            \centering
            \includegraphics[width=0.6\textwidth]{./assets/fig_6.png}

            The intercept at $X_B = 1$ is around $\SI{-3032}{\joule\per\mole}$.

            \begin{align*}
              \Delta G &= RT\ln a \\
              a &= \exp\left(\frac{\Delta G}{RT}\right) \\
              a &=
              \exp\left(\frac{\SI{-3032}{\joule\per\mole}}{(\SI{8.314}{\joule\per\mole\kelvin})(\SI{500}{\kelvin})}\right)
              \\
              \Aboxed{a &= 0.482}
            \end{align*}
          }
      \end{enumerate}

  \end{enumerate}

  \pagebreak

  \section*{Supporting code:}
  \inputminted{julia}{./calculations/src/calculations.jl}

  \end{document}
